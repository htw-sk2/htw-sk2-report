%% Rechtliche Begutachtung
% Aufgabe 1
\section{Rechtliche Begutachtung}

In diesem Kapitel erfolgt die rechtliche Begutachtung zum Planungsstand der \num{20} zu installierenden griechischen Kleinwindanlagen in Deutschland. Dabei lässt sich durch unterschiedliche Anschlussbedingungen zwischen drei Situationen unterscheiden:

\begin{enumerate}
	\item Anschluss am Mittelspannungsnetz in bestehenden Mischparks mit einer Netzanschlussleistung von mindestens \SI{1}{\mw}
	\item Anschluss am Mittelspannungsnetz ohne Integration in bestehende Strukturen
	\item Anschluss in Inselnetzen
\end{enumerate}

Neben den Anschlussbedingungen, sind vor allem die technischen Details der einzelnen \glspl{EZE} dafür entscheidend, welche Netzanschlusszertifizierung zum tragen kommt. In \autoref{tab:TechDetails} finden sich die wichtigsten technischen Details der \glspl{EZE}.

{
\renewcommand{\arraystretch}{1.2}% grßerer Zeilenabstand
\sisetup{
	range-phrase=~{--}~,% Gedankenstrich statt "bis" bei SIrange
}
\begin{table}[H]
	\begin{center}
		\caption{Technische Daten der griechischen Kleinwindanlagen}
		\begin{tabu} to \textwidth {X X[r]}
			\hline
			\multicolumn{2}{c}{Technische Daten}		\\ \hline
			Nennleistung		&	\SI{100}{\kw}		\\
			Nabenhöhe			&	\SI{50}{\meter}		\\ \hline
		\end{tabu}
		\label{tab:TechDetails}
	\end{center}
	\vspace{-3mm}%Put here to reduce too much white space after your table
\end{table}
}

\subsection{Nachweisverfahren für Prototypen}

Mitte \num{2019} wurde bereits eine \gls{EZE} am deutschen Niederspannungsnetz angeschlossen. Dabei kam das Nachweisverfahren für Prototypen gemäß \texttt{VDE-AR-N \num{4110}:\num{2018}} zum Einsatz. Als Prototyp liegt für die \gls{EZE} noch kein Einheitenzertifikat vor. Jedoch müssen für \glspl{EZA} mit Prototypen bereits folgende Unterlagen beim Netzbetreiber eingereicht werden:

\begin{itemize}
	\item Anschlussanmeldung (\texttt{E.1})
	\item Datenblatt der \gls{EZA} (\texttt{E.8})
	\item Datenblatt der elektrischen Eigenschaften der \gls{EZE}
	\item Abschätzung der elektrischen Eigenschaften der \gls{EZE}
	\item Elektroplanung der \gls{EZA} (Lastfluss-Berechnung, Wirkleistungssteuerung, statische Spannungshaltung, Schutzkonzept, Abschätzung der Netzrückwirkung)
	\item Prototypenbestätigung
\end{itemize}

Die Prototypenbestätigung steht bei Prototypen stellvertretend für das Einheitenzertifikat. Das Einheitenzertifikat muss innerhalb von zwei Jahren nach der Inbetriebnahme des Prototypen ausgestellt werden. Damit alle Anlagen \num{2021} an das Netz gehen können, muss sich darum bemüht werden, die \gls{EZE} baldmöglichtst durch eine Zertfizierungsstelle auf die Einhaltung der Richtlinien und Grid Codes prüfen zu lassen. Das Einheitenzertifikat wird für die Inbetriebnahme der restlichen \glspl{EZA} benötigt. Nach der Ausstellung des Einheitenzertifikats, muss innerhalb eines Jahres das Anlagenzertifikat nachgereicht werden und anschließend die Konformitätserklärung eingeholt werden. \cite{MOEGH2020}

\subsection{Netzanschlusszertifizierung}

In \autoref{tab:EZAVerfahren} finden sich die je nach Anschlussbedingungen anzuwendenden Anschlussregelungen und Zertifizierungen. Eine Begründung für die Einordnung findet sich in den jeweiligen Unterkapiteln.

{
\renewcommand{\arraystretch}{1.2}% grßerer Zeilenabstand
\sisetup{
	range-phrase=~{--}~,% Gedankenstrich statt "bis" bei SIrange
}
\begin{table}[H]
	\begin{center}
		\caption{Anzuwendende Anschlussregelung und Zertifizierung der Erzeugungsanlagen}
		\begin{tabu} to \textwidth {X[2] X X}
			\hline
			Anschlussbedingungen					&	Anschlussregelung				&	Zertifizierung		\\ \hline
			Mischpark im Mittelspannungsnetz		&	\texttt{VDE-AR-N \num{4110}}	&	Anlagenzertifikat A	\\
			Einzelanschluss im Mittelspannungsnetz	&	\texttt{VDE-AR-N \num{4105}}	&	Einheitenzertifikat	\\
			Inselnetzanschluss						&	{--}							&	{--}				\\ \hline
		\end{tabu}
		\label{tab:EZAVerfahren}
	\end{center}
	\vspace{-3mm}%Put here to reduce too much white space after your table
\end{table}
}

\subsubsection{Anschluss in einem Mischpark im Mittelspannungsnetz}

Für alle \glspl{EZA} mit Mittelspannungsanschluss mit einer Anschlussleistung von \SI{\geq~950}{\kw} muss eine Zertifizierung des Anlagenzertifikats A nach \texttt{VDE-AR-N \num{4110}} erfolgen. Durch die Vorlage des Zertifikats beim Netzbetreiber, sichert sich der Anlagenbetreiber den Netzanschluss und Vergütung. Dabei sind folgende Dokumente für die Erstellung des Anlagenzertifikats A bereits durch das Nachweisverfahren für Prototypen vorhanden bzw. werden noch innerhalb des Verfahrens ausgestellt:

\begin{itemize}
	\item Datenblatt der \gls{EZA} (\texttt{E.8})
	\item Deckblätter der Einheitenzertifikate
	\item  Elektroplanung der \gls{EZA} (Lastfluss-Berechnung, Wirkleistungssteuerung, statische Spannungshaltung, Schutzkonzept, Abschätzung der Netzrückwirkung)
\end{itemize}

Weiterhin müssen ebenfalls die technischen Daten der Bestands-\gls{EZE} vorgelegt werden. Außerdem müssen zusätzlich folgende Dokumente angefertigt und vorgelegt werden:

\begin{itemize}
	\item Netzrückwirkungen der Verbrauchsgeräte (\texttt{E.2})
	\item Einphasiger Übersichtsschaltplan
	\item Regelungskonzept inkl. Kommunikationsplan
	\item Angabe der geplanten Stufenstellerposition der Maschinentransformatoren
	\item Lageplan inkl. Koordinaten der Erzegungseinheiten
\end{itemize}

Um den Anschluss im Jahr \num{2021} möglichst zu gewährleisten, sollte frühzeitig Kontakt mit einer Zertifizierungsstelle aufgenommen werden und die Anfertigung aller Unterlagen zum Zeitpunkt der Ausstellung des Einheitenzertifikat im Prototypenverfahren abgeschlossen sein. \cite{MOEG2020}

\subsubsection{Einzelanschluss im Mittelspannungsnetz}

Im Gegensatz zum Anschluss der \gls{EZE} in bestehenden Mischparks, kann bei Einzelanschluss von \glspl{EZA} im Mittelspannungsnetz mit einer Anschlussleistung von \SI{<~135}{\kw} auf die Zertifizierung nach \texttt{VDE-AR-N \num{4105}} zurückgegriffen werden. In diesem Fall ist ein Einheitenzertifikat ausreichend, wobei die Mittelspannungsschaltanlage weiterhin nach \texttt{VDE-AR-N \num{4110}} zertifiziert werden muss. Somit muss in diesen Fällen nur auf die Ausstellung Einheitenzertifikates des Prototypenverfahrens gewartet werden.

\subsubsection{Anschluss im Inselnetz}

Für einen Anschluss im Inselnetz ist keine Zertifizierung nach \texttt{VDE-AR-N} nötig. Somit muss dem Netzbetreiber gegenüber nur der Inselnetzstatus nachgewiesen werden und es müssen keine Zertifikate ausgestellt werden.